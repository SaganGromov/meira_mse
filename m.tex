\documentclass{article}
\usepackage{amsmath, amssymb, amsfonts}
\usepackage[a4paper,margin=0.5in]{geometry}

\begin{document}

Let $\mathcal{M}$ be a Riemannian manifold. If you have a coordinate system, say $\xi=\left(x^{1},\ldots,x^{n}\right)$ centered at $p$ in $\mathcal{M}$, then a basis for the tangent space $T_p \mathcal{M}$ is given by $\left\{\frac{\partial}{\partial x^{1}}, \ldots, \frac{\partial}{\partial x^{n}}\right\}$, which we simply write as $\left\{\partial_1, \ldots, \partial_n \right\}$. The Basis Theorem (following the book \textit{Semi-Riemannian Geometry} by Barrett O'Neill) states that any vector $v$ tangent to $\mathcal{M}$ at $p$ can be written as 
\[
v = \sum_{i=1}^{n} v(x^{i}) \partial_i.
\]

For your example, let $\xi = (x,y,z)$ be the natural coordinate system in $\mathbb{R}^{3}$ with $\left\{\partial_x, \partial_y, \partial_z\right\}$ as the associated tangent basis, and let $\eta = (a,\phi,h)$ be the natural cylindrical coordinate system in $\mathbb{R}^{3} \setminus H$, where $H$ is the hyperplane $x \geq 0, y=0$. Now, define $x = a \cos\phi$, $y = a \sin\phi$, and $z = h$. Notice that 
\[
\left\{\frac{\partial}{\partial a}, \frac{\partial}{\partial \phi}, \frac{\partial}{\partial h}\right\} \stackrel{\text{notation}}{=} \left\{\partial_a, \partial_\phi, \partial_h\right\}
\]
is composed of tangent vectors, to which we can apply the Basis Theorem:
\begin{align*}
   \partial_a &= \partial_a(x)\partial_x + \partial_a(y)\partial_y + \partial_a(z)\partial_h \\
              &= \cos(\phi)\partial_x + \sin(\phi)\partial_y,
\end{align*}
\begin{align*}
   \partial_\phi &= \partial_\phi(x)\partial_x + \partial_\phi(y)\partial_y + \partial_\phi(z)\partial_h \\
              &= -a\sin(\phi)\partial_x + a\cos(\phi)\partial_y,
\end{align*}
and
\begin{align*}
     \partial_h &= \partial_h(x)\partial_x + \partial_h(y)\partial_y + \partial_h(z)\partial_h \\
                &= \partial_h.
\end{align*}

Now, recall that every metric $g$ is a $(0,2)$-type tensor, which in particular implies its local expression is a linear combination of (tensor) products of one forms (whose coefficients are exactly the components of the metric in the associated coordinate system). When evaluated at a point, it's an inner product, which when evaluated at two vectors returns a real number. To sum it up, if $\xi$ is the aforementioned coordinate system, we can write
\[
g = \sum g_{ij} \mathrm{d}x^{i} \otimes \mathrm{d} x^{j},
\]
where $\{\mathrm{d}x^{1}, \ldots, \mathrm{d}x^{n}\}$ is the local frame for the contangent space, determined by the frame $\{ \partial_1, \ldots, \partial_n \}$, and $g_{ij} = g\left(\partial_i,\partial_j \right)$. Thus, the line element is
\[
\mathrm{d}s^{2} = \sum  g_{ij}\mathrm{d}x^{i} \otimes \mathrm{d} x^{j}
\]

For your example, from $\left\{\partial_a, \partial_\phi, \partial_h\right\}$, we have $\{\mathrm{d}a, \mathrm{d}\phi, \mathrm{d}h\}$ as the cotangent basis. Set $a = y^{1}$, $\phi = y^{2}$, and $h = y^{3}$. The line element in the $\eta$ coordinate system is written as 
\[
\mathrm{d}s^{2} = \sum_{i,j = 1}^{3} g_{ij} \mathrm{d}y^{i} \otimes \mathrm{d}y^{j}.
\]

The last step is to compute the $g_{ij}$:
\[
\begin{aligned}
    g_{11} &= g(\partial_a, \partial_a) = \cos^{2}\phi + \sin^{2}\phi = 1, \\
    g_{22} &= g(\partial_\phi, \partial_\phi) = a^{2}\sin^{2}\phi + a^{2}\cos^{2}\phi = a^{2}, \\
    g_{33} &= g(\partial_h, \partial_h) = 1,
\end{aligned}
\]
while $g_{ij} = 0$ for $i \neq j$, because $g(\partial_a,\partial_\phi) = -a\sin\phi\cos\phi + a\sin\phi\cos\phi = 0$ and $g(.,\partial_h) = 0$ as other vectors have $0$ as a component in the $\partial_h$ direction.

Finally, we obtain
\[
    \mathrm{d}s^{2} = g_{11} \ \mathrm{d}a \otimes \mathrm{d}a + g_{22} \ \mathrm{d}\phi \otimes \mathrm{d}\phi + g_{33} \mathrm{d}h \otimes \mathrm{d}h = \mathrm{d}a^{2} + a^{2}\mathrm{d}\phi^{2} + \mathrm{d}h^{2},
\]
as desired.

\end{document}
